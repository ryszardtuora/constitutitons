
\documentclass{book}
\usepackage[utf8]{inputenc}
\usepackage[polish]{babel}
\usepackage{polski}
\begin{document}
\pagestyle{plain}
\pagenumbering{gobble}

\title{USTAWA RZĄDOWA}
\date{z dnia 3-go maja 1791 roku}
\maketitle

\vspace*{\fill}
\begin{Large}
\begin{centering}
\textit{Stanisław August z Bożej łaski i woli Narodu Król Polski, Wielki Książe Litewski, Ruski, Pruski, Mazowiecki, Żmudzki, Kijowski, Wołyński, Podolski, Inflancki,  Smoleński, Siewierski i Czerniechowski, wraz z stanami skonfederowanemi, w liczbie podwójnej Naród Polski reprezentującemi, uznając, iż los nas wszystkich  od ugruntowania i wydoskonalenia konstytucji narodowej jedynie zawisł, długim doświadczeniem poznawszy zadawnione rządu naszego wady, a chcąc korzystać z pory, w jakiej się Europa znajduje, i z tej dogorywającej chwili, która nas samym sobie wróciła, wolni od hańbiących obcej przemocy nakazów, ceniąc drożej nad życie,  nad szczęśliwość osobistą egzystencyją polityczną, niepodległość zewnętrzną i wolność wewnętrzną narodu, którego los w ręce nasze jest powierzony,  chcąc oraz na błogosławieństwo, na wdzięczność współczesnych i przyszłych pokoleń zasłużyć, mimo przeszkód, które w nas namiętności sprawować mogą,  dla dobra powszechnego, dla ugruntowania wolności, dla ocalenia ojczyzny naszej i jej granic, z największą stałością ducha niniejszą konstytucję uchwalamy  i tę całkowicie za świętą, za niewzruszoną deklarujemy, dopóki by naród w czasie,  prawem przepisanym, wyraźną  wolą swoją nie uznał potrzeby odmienienia w niej jakiego artykułu.  Do której to konstytucji dalsze ustawy sejmu teraźniejszego we wszystkiem stosować się mają.}
\end{centering}
\end{Large}

\newpage
\pagenumbering{arabic} 

 
\section*{I. Religia panująca}

 Religią narodową panującą jest i będzie wiara święta rzymska katolicka ze wszystkiemi jej prawami.  Przejście od wiary panującej do jakiegokolwiek wyznania jest zabronione pod karami apostazji \footnote{Apostazja - wyklęcia}. Że zaś ta sama wiara święta przykazuje nam kochać bliźnich naszych, przeto wszystkim ludziom, jakiegokolwiek bądź wyznania,  pokój w wierze i opiekę rządową winniśmy i dlatego wszelkich obrządków i religij wolność w krajach polskich, podług ustaw krajowych, warujemy.  

 
\section*{II. Szlachta ziemianie}

 Szanując pamięć przodków naszych jako fundatorów rządu Wolnego stanowi szlacheckiemu wszelkie swobody, wolności, prerogatywy, pierwszeństwa w życiu prywatnem i publicznem najuroczyściej zapewniamy, szczególniej zaś prawa, statuta i przywileje temu stanowi od Kazimierza Wielkiego, Ludwika Węgierskiego, Władysława Jagiełły i Witolda, brata jego, wielkiego księcia Litewskiego,  niemniej od Władysława i Kazimierza Jagiellończyków, od Jana Alberta \footnote{Jana Alberta - Jana Olbrachta},  Aleksandra i Zygmunta Pierwszego braci, od Zygmunta Augusta ostatniego z linji jagiellońskiej sprawiedliwie i prawnie nadane, utwierdzamy, zapewniamy i za niewzruszone uznajemy.  Godność stanu szlacheckiego w Polsce za równą wszelkim stopniom szlachectwa gdziekolwiek używanym przyznajemy.  Wszystką szlachtę równymi być między sobą uznajemy, nie tylko co do starania się o urzędy i sprawowanie posług ojczyźnie, honor, sławę, pożytek przynoszących,  ale oraz co do równego używania przywilejów i prerogatyw, stanowi szlacheckiemu służących, nade wszystko zaś prawa bezpieczeństwa osobistego, wolności osobistej  i własności gruntowej i ruchomej, tak jak od wieków każdemu służyły, świątobliwie, nienaruszenie zachowane mieć chcemy i zachowujemy,  zaręczając najuroczyściej, iż przeciwko własności czyjejkolwiek żadnej odmiany lub ekscepcji \footnote{Ekscepcji - wyjątków}  w prawie nie dopuścimy, owszem, najwyższa władza krajowa i rząd przez nią ustanowiony żadnych pretensyj pod pretekstem iurium regalium\footnote{Iurium regalium - wyłącznych praw królewskich}  i jakimkolwiek innym pozorem do własności obywatelskich bądź w części, bądź w całości rościć sobie nie będzie.  Dlaczego bezpieczeństwo osobiste i wszelką własność, komukolwiek z prawa przynależną, jako prawdziwy społeczności węzeł, jako źrenicę wolności obywatelskiej szanujemy,  zabezpieczamy, utwierdzamy i aby na potomne czasy szanowane, ubezpieczone i nienaruszone zostawały mieć chcemy.  Szlachtę za najpierwszych obrońców wolności i niniejszej konstytucji uznajemy.  Każdego szlachcica cnocie, obywatelstwu i honorowi jej świętość do szanowania, jej trwałość do strzeżenia poruczamy, jako jedyną twierdzę ojczyzny i swobód naszych. 

 
\section*{III. Miasta i mieszczanie}

 Prawo na teraźniejszym sejmie zapadłe pod tytułem: \textit{Miasta nasze królewskie wolne w państwach Rzeczypospolitej} w zupełności utrzymane mieć chcemy i za część niniejszej konstytucji deklarujemy, jako prawo wolnej szlachcie polskiej, dla bezpieczeństwa ich swobód i całości wspólnej ojczyzny, nową,  prawdziwą i skuteczną dające siłę. 

 
\section*{IV. Chłopi włościanie}

Lud rolniczy, spod którego ręki płynie najobfitsze bogactw krajowych źródło, który najliczniejszą w narodzie stanowi ludność,
a zatem najdzielniejszą kraju siłę, tak przez sprawiedliwość, ludzkość i obowiązki chrześcijańskie, jako i przez własny nasz interes dobrze zrozumiany, pod opiekę prawa
i rządu krajowego przyjmujemy, stanowiąc, iż odtąd jakiebykolwiek swobody, nadania lub umowy dziedzice z włościanami dóbr swoich autentycznie ułożyli,
czyliby te swobody, nadania i umowy były z gromadami, czyli też z każdym osobno wsi mieszkańcem zrobione, będą stanowić wspólny i wzajemny obowiązek, podług
rzetelnego znaczenia warunków i opisu zawartego w takowych nadaniach i umowach, pod opiekę rządu krajowego podpadający.
Układy takowe i wynikające z nich obowiązki, przez jednego właściciela gruntu dobrowolnie przyjęte, nie tylko jego samego, ale i następców jego lub prawa nabywców tak wiązać będą,
iż ich nigdy samowolnie odmieniać nie będą mocni.
Nawzajem włościanie jakiejkolwiek bądź majętności od dobrowolnych umów, przyjętych nadań i z nimi złącznych powinności usuwać się inaczej nie będą mogli,
tylko w takim sposobie i z takimi warunkami, jak w opisach tychże umów postanowione mieli, które, czy na wieczność, czyli do czasu przyjęte, ściśle ich obowiązywać będą.
Zawarowawszy tym sposobem dziedziców przy wszelkich pożytkach od włościan im należących, a chcąc jak najskuteczniej zachęcić pomnożenie ludności krajowej,
ogłaszamy wolność zupełną dla wszystkich ludzi, tak nowo przybywających, jak i tych, którzy by, pierwej z kraju oddaliwszy się, teraz do ojczyzny powrócić chcieli,
tak dalece, iż każdy człowiek do państw Rzeczypospolitej nowo z którejkolwiek strony przybyły lub powracający, jak tylko stanie nogą na ziemi Polskiej,
wolnym jest zupełnie użyć przemysłu swojego, jak i gdzie chce, wolny jest czynić umowy na osiadłość, robociznę lub czynsze, jak i dopóki się umówi,
wolny jest osiadać w mieście lub na wsiach, wolny jest mieszkać w Polsce lub do kraju, do którego zechce, powrócić, uczyniwszy zadosyć obowiązkom, które dobrowolnie na siebie przyjął.
 
\section*{V. Rząd czyli oznaczenie władz publicznych}

 Wszelka władza społeczności ludzkiej początek swój bierze z woli narodu. Aby więc całość państw, wolność obywatelską i porządek społeczności w równej wadze na zawsze zostawały, trzy władze rząd narodu Polskiego składać powinny  i z woli prawa niniejszego na zawsze składać będą, to jest władza prawodawcza w stanach zgromadzonych, władza najwyższa wykonawcza w królu i straży i władza sądownicza w jurysdykcjach, na ten koniec ustanowionych, lub ustanowić się mających. 

 
\section*{VI. Sejm czyli władza prawodawcza}

 Sejm czyli stany zgromadzone na dwie izby dzielić się będą: na izbę poselską i na izbę senatorską pod prezydencją króla. Izba poselska, jako wyobrażenie i skład wszechwładztwa narodowego, będzie świątynią prawodawstwa. Przeto w izbie poselskiej najpierwej decydowane będą wszystkie projekty:  
\begin{itemize}
\item 1-mo co do praw ogólnych, to jest konstytucyjnych, cywilnych, kryminalnych i do ustanowienia wieczystych podatków, w których to materjach propozycje od tronu województwom, ziemiom i powiatom do roztrząśnienia podane, a przez instrukcje do izby przychodzące, najpierwsze do decyzji wzięte być mają; 
\item 2-do co do uchwał sejmowych, to jest poborów doczesnych, stopnia monety, zaciągania długu publicznego, nobilitacyj \footnote{Nobilitacja - nadawanie tytułów i praw szlacheckich} i innych nagród przypadkowych, rozkładu wydatków publicznych, ordynaryjnych\footnote{Ordynaryjnych - zwykłych, zazwyczaj uchwalanych} i ekstraordynaryjnych, \footnote{Ekstraordynaryjnych - wyjątkowych}, wojny, pokoju, ostatecznej ratyfikacji traktatów związkowych i handlowych, wszelkich dyplomatycznych aktów i umów, do prawa narodów ściągających się,  kwitowania magistratur wykonawczych\footnote{Magistratura wykonawcza - postanowienie sądowe} i tym podobnych zdarzeń, głównym narodowym potrzebom odpowiadających, w których to materjach propozycje od tronu, prosto do izby poselskiej przychodzić mające,  pierwszeństwo w prowadzeniu mieć będą. 
\end{itemize}

 Izby senatorskiej, złożonej z biskupów, wojewodów, kasztelanów i ministrów pod prezydencją króla, mającego prawo raz dać votum\footnote{Votum - głos} swoje, drugi raz paritatem\footnote{Paritatem - w głosowaniu równa ilość głosów \textit{za} i \textit{przeciw}} rozwiązywać osobiście lub nadesłaniem zdania swego do tejże izby, obowiązkiem jest 
\begin{itemize}
\item 1-mo każde prawo, które po przejściu formalnem w izbie poselskiej do senatu natychmiast przesłane być powinno, przyjąć lub wstrzymać do dalszej narodu deliberacji \footnote{Deliberacja - rozważanie, naradzanie się} opisaną w prawie większością głosów. Przyjęcie moc i świętość prawa nadawać będzie. Wstrzymanie zaś zawiesi tylko prawo do przyszłego ordynaryjnego sejmu, na którym gdy powtórna nastąpi zgoda, prawo zawieszone od senatu przyjętem być musi; 
\item 2-do każdą uchwałę sejmową w materjach wyżej wyliczonych, którą izba poselska senatowi przysłać natychmiast powinna,  wraz z tąż izbą poselską większością głosów decydować, a złączona izb obydwóch większość, podług prawa opisana, będzie wyrokiem i wolą stanów.  
\end{itemize}
Warujemy, iż senatorowie i ministrowie w objektach sprawowania się z urzędowania swego bądź w straży, bądź w komisji, votum decisivum\footnote{Votum decisivum - z głosem decydującym} w sejmie nie będą mieli i tylko zasiadać wtenczas w senacie mają dla dania eksplikacyj\footnote{Eksplikacja - wyjaśnienie} na żądanie sejmu. Sejm zawsze gotowym będzie: prawodawczy i ordynaryjny. Rozpoczynać się ma co dwa lata, trwać zaś będzie podług opisu prawa o sejmach.  Gotowy, w potrzebach nagłych zwołany, stanowić ma o tej tylko materji, do której zwołanym będzie, lub o potrzebie, po czasie zwołania przypadłej.  Prawo żadne na tym ordynaryjnym sejmie, na którym ustanowione było, znoszonym być nie może. Komplet sejmu składać się będzie z liczby osób, niższym prawem opisanej, tak w izbie poselskiej, jako i w izbie senatorskiej. \textit{Prawo o sejmikach}, na teraźniejszym sejmie ustanowione, jako najistotniejszą zasadę wolności obywatelskiej, uroczyście zabezpieczamy. 

 Jako zaś prawodawstwo sprawowane być nie może przez wszystkich i naród wyręcza się w tej mierze przez reprezentantów czyli posłów swoich dobrowolnie wybranych, przeto stanowimy, iż posłowie na sejmikach obrani w prawodawstwie i ogólnych narodu potrzebach podług niniejszej konstytucji uważani być mają jako reprezentanci całego narodu, będąc składem ufności powszechnej. Wszystko i wszędzie większością głosów udecydowane być powinno; przeto liberum veto, konfederacje wszelkiego gatunku i sejmy konfederackie, jako duchowi niniejszej konstytucji przeciwne, rząd obalające, społeczność niszczące, na zawsze znosimy. Zapobiegając z jednej strony gwałtownym i częstym odmianom konstytucji narodowej, z drugiej uznając potrzebę wydoskonalenia onej, po doświadczeniu jej skutków,   co do pomyślności publicznej, porę i czas rewizji i poprawę konstytucji co lat dwadzieścia pięć naznaczamy.  Chcąc mieć takowy sejm konstytucyjny ekstraordynaryjnym podług osobnego o nim prawa opisu. 

 
\section*{VII. Król, władza wykonawcza}

 Żaden rząd najdoskonalszy bez dzielnej władzy wykonawczej stać nie może. Szczęśliwość narodów od praw sprawiedliwych, praw skutek od ich wykonania zależy. Doświadczenie nauczyło, że zaniedbanie tej części rządu nieszczęściami napełniło Polskę. Zawarowawszy przeto wolnemu narodowi Polskiemu władzę praw sobie stanowienia i moc baczności nad wszelką wykonawczą władzą, oraz wybierania urzędników do magistratur, władzę najwyższego wykonywania praw królowi w radzie jego oddajemy, która to rada Strażą Praw zwać się będzie. Władza wykonawcza do pilnowania praw i onych pełnienia ściśle jest obowiązaną. Tam czynną z siebie będzie, gdzie prawa dozwalają, gdzie prawa potrzebują dozoru, egzekucji, a nawet silnej pomocy.  Posłuszeństwo należy jej się zawsze od wszystkich magistratur, moc przynaglenia nieposłuszne i zaniedbujące swe obowiązki magistratury w jej ręku zostawiamy.  Władza wykonawcza nie będzie mogła praw stanowić ani tłumaczyć, podatków i poborów pod jakimkolwiek imieniem nakładać, długów publicznych zaciągać, rozkładu dochodów skarbowych przez sejm zrobionego odmieniać,  wojny wydawać, pokoju ani traktatu  i żadnego aktu dyplomatycznego definitive\footnote{Definitive - ostatecznie} zawierać. Wolno jej tylko będzie tymczasowe z zagranicznemi prowadzić negocjacje oraz tymczasowe i potoczne dla bezpieczeństwa i spokojności kraju wynikające potrzeby załatwiać,  o których najbliższemu zgromadzeniu sejmowemu donieść winna.  Tron Polski elekcyjnym przez familje\footnote{Familie - rodziny, rody, dynastie} mieć na zawsze chcemy i stanowimy.  Doznane klęski bezkrólewiów, perjodycznie rząd wywracających, powinność ubezpieczenia losu każdego mieszkańca ziemi Polskiej i zamknięcie na zawsze drogi wpływów mocarstw zagranicznych, pamięć świetności i szczęścia ojczyzny naszej, za czasów familij ciągle panujących, potrzeba odwrócenia od ambicji tronu obcych i możnych Polaków,  zwrócenia do jednomyślnego wolności narodowej pielęgnowania, wskazały roztropności naszej oddanie tronu polskiego prawem następstwa.  Stanowimy przeto, iż po życiu, jakiego nam dobroć Boska pozwoli, elektor dzisiejszy Saski w Polsce królować będzie.  Dynastja przyszłych królów polskich zacznie się na osobie Fryderyka Augusta, dzisiejszego elektora Saskiego, którego sukcesorem de lumbis\footnote{De lumbis - legalnie zrodzonym, z prawego łoża} z płci męskiej tron Polski przeznaczamy. Najstarszy syn króla panującego po ojcu na tron następować ma.  Gdyby zaś dzisiejszy elektor Saski nie miał potomstwa płci męskiej, tedy mąż przez elektora, za zgodą stanów zgromadzonych córce jego dobrany, zaczynać ma linią następstwa płci męskiej do tronu Polskiego.  Dla czego Marię Augustę Nepomucenę, córkę elektora, za infantkę Polską deklarujemy, zachowując przy narodzie prawo, żadnej preskrypcji\footnote{Preskrypcja - ograniczenie} podpadać nie mogące, wybrania do tronu drugiego domu po wygaśnięciu pierwszego. 

 Każdy król, wstępując na tron, wykona przysięgę Bogu i Narodowi na zachowanie konstytucji niniejszej, na pacta conventa\footnote{Pacta conventa - układ pomiędzy nowoobranym królem elekcyjnym a narodem reprezentowanym przez sejm i senat określający wzajemne prawa i obowiązki.  Bez ich podpisania król-elekt nie mógł być koronowany.  Obowiązywały od pierwszej elekcji - Henryka Walezego. } , które ułożone będą z dzisiejszym elektorem Saskim, jako przeznaczonym do tronu i które, tak jak dawne, wiązać go będą.  Osoba króla jest święta i bezpieczna od wszystkiego.  Nic sam przez się nie czyniący, za nic w odpowiedzi narodowi być nie może.  Nie samowładcą, ale ojcem i głową narodu być powinien i tym go prawo i konstytucja niniejsza być uznaje i deklaruje.  Dochody tak, jak będą w paktach konwentach opisane i prerogatywy tronowi właściwe, niniejszą konstytucję dla przyszłego elekta zawarowane, tkniętymi być nie będą mogły.  Wszystkie akta publiczne, trybunały, sądy, magistratury, monety, stemple pod królewskim powinny iść imieniem.  Król, którego mu wszelka moc dobrze czynienia zostawiona być powinna, mieć będzie ius agratiandi\footnote{Ius agratiandi - prawo łaski} na śmierć skazanych, prócz in criminibus status\footnote{In criminibus status - w przestępstwach zbrodni stanu}.  Do króla rozrządzenie najwyższe siłami zbrojnymi krajowymi w czasie wojny i nominowanie komendantów wojska należeć będzie, z wolną atoli ich odmianą za wolą narodu.  Patentować oficerów i mianować urzędniki podług prawa niższego opisu, nominować biskupów i senatorów podług opisu tegoż prawa,  oraz ministrów, jako urzędników pierwszych władzy wykonawczej, jego będzie obowiązkiem.  

 Straż, czyli rada królewska, do dozoru całości i egzekucji praw królowi dodana, składać się będzie: 
\begin{itemize}
\item 1-mo z prymasai, jako głowy duchowieństwa polskiego  i jako prezesa komisji edukacyjnej, mogącego być wyręczonym w straży przez pierwszego ex ordine\footnote{Ex ordine - z porządku; tu: w hierarchii}  biskupa, którzy rezolucji podpisywać nie mogą;  
\item 2-do z pięciu ministrów, to jest ministra policji, ministra pieczęci, ministra belli, \footnote{Belli - wojny}, ministra skarbu, ministra pieczęci do spraw zagranicznych; 
\item 3-tio, z dwuch sekretarzów, z których jeden protokół straży, drugi protokół spraw zagranicznych trzymać będą, obydwa bez votum decydującego.  
\end{itemize}
Następca tronu z małoletności wyszedłszy i przysięgę na konstytucję wykonawszy, na wszystkich straży posiedzeniach, lecz bez głosu, przytomnym być może.  Marszałek sejmowy, jako na dwa lata wybrany, wchodzić będzie w liczbę zasiadających w straży, bez wdawania się w jej rezolucje, jedynie dla zwołania sejmu gotowego. W takiem zdarzeniu gdyby on uznał, w przypadkach koniecznego zwołania sejmu gotowego wymagających, rzetelną potrzebę, a król go zwołać wzbraniał się,  tedy tenże marszałek do posłów i senatorów wydać powinien listy okólne, zwołując onych na sejm gotowy i powody zwołania tego wyrażając. Przypadki zaś do koniecznego zwołania sejmu są tylko następujące:  
\begin{itemize}
\item 1-mo w gwałtownej potrzebie do prawa narodu ściągającej się, a szczególniej w przypadku wojny ościennej; 
\item 2-do w przypadku wewnętrznego zamieszania grożącego rewolucją kraju lub kolizją między magistraturami;  
\item 3-tio w widocznym powszechnego głodu niebezpieczeństwie; 
\item 4-to w osierociałym stanie ojczyzny przez śmierć króla lub w niebezpiecznej jego chorobie. 
\end{itemize}
Wszystkie rezolucje w straży roztrząsane będą przez skład wyżej wymieniony.  Decyzja królewska po wysłuchanych wszystkich zdaniach przeważać powinna, aby jedna była w wykonaniu prawa wola.  Przeto każda ze straży rezolucja pod imieniem królewskim i z podpisem ręki jego wychodzić będzie, powinna jednak być podpisana także przez jednego z ministrów, zasiadających w straży, i tak podpisana do posłuszeństwa wiązać będzie,  i dopełnioną być ma przez komisje, bądź przez jakiekolwiek magistratury wykonawcze, w tych jednak szczególnie materjach, które wyraźnie niniejszym prawem wyłączone nie są.  W przypadku, gdyby żaden z ministrów zasiadających decyzji podpisać nie chciał, król odstąpi od tej decyzji, a gdyby przy niej upierał się, marszałek sejmowy,  w tym przypadku, upraszać będzie o zwołanie sejmu gotowego, i jeżeli król spóźniać będzie zwołanie, marszałek to wykonać powinien. 

 Jako nominowanie wszystkich ministrów, tak i wezwanie z nich jednego od każdego administracji wydziału do rady swojej czyli straży króla jest prawem.  Wezwanie do ministra do zasiadania w straży na lat dwa będzie, z wolnem onego nadal przez króla potwierdzeniem.  Ministrowie, do straży wezwani, w komisjach zasiadać nie mają.  W przypadku zaś, gdyby większość dwuch trzecich wotów sekretnych obydwuch izb złączonych na sejmie ministra bądź w straży, bądź w urzędzie odmiany żądała,  król natychmiast na jego miejsce innego nominować powinien.  Chcąc, aby Straż Praw narodowych obowiązana była do ścisłej odpowiedzi narodowi za wszelkie onych przestępstwa, stanowimy, iż, gdy ministrowie będą oskarżeni przez deputację,  do egzaminowania ich czynności wyznaczoną, o przestępstwo prawa, odpowiadać mają z osób i majątków swoich.  W wszelkich takowych oskarżeniach stany zgromadzone prostą większością wotów izb złączonych odesłać obwinionych ministrów mają do sądów sejmowych po sprawiedliwe i wyrównywające przestępstwu ich ukaranie,  lub przy dowiedzionej niewinności od sprawy i kary uwolnione.  Dla porządnego władzy wykonawczej dopełnienia, ustanawiamy oddzielne komisje, mające związek ze strażą i obowiązane do posłuszeństwa tejże straży.  Komisarze do nich wybierani będą przez sejm dla sprawowania urzędów swoich w przeciągu czasu, prawem opisanego.  Komisje te są: 
\begin{itemize}
\item 1-mo edukacji; 
\item 2-o policji; 
\item 3-tio wojska; 
\item 4-to skarbu. 
\end{itemize}
Komisje porządkowe wojewódzkie na tym sejmie ustanowione, równie do dozoru straży należące, dobierać będą rozkazy przez wyżej wspomniane pośrednicze komisje respective\footnote{Respective - przestrzegając, odnośnie do} co do objektów każdej z nich władzy i obowiązków.   

 
\section*{VIII. Władza sądownicza}

 Władza sądownicza nie może być wykonywaną ani przez władzę prawodawczą, ani przez króla, lecz przez magistratury, na ten koniec ustanowione i wybierane. Powinna zaś być tak do miejsc przywiązaną, żeby każdy człowiek bliską dla siebie znalazł sprawiedliwość, żeby przestępny widział wszędzie groźną nad sobą rękę krajowego rządu.  
\begin{itemize}
\item 1-mo. Ustanawiamy przeto Sądy pierwszej instancji dla każdego województwa, ziemi i powiatu, do których sędziowie wybierani będą na sejmikach.  Sądy pierwszej instancji będą zawsze gotowe i czuwające na oddanie sprawiedliwości tym, którzy jej potrzebują.  Od tych sądów iść będzie apelacja na trybunały główne, dla każdej prowincji być mające, złożone równie z osób na sejmikach wybranych.  I te sądy tak pierwszej, jako i ostatniej instancji będą sądami ziemiańskiemi dla szlachty i wszystkich właścicieli ziemskich z kimkolwiek   in causis iuris et facti\footnote{In causis iuris et facti - w sprawach zgodnie z prawem i procedurą}. 
\item 2-do. Jurysdykcje zaś sądowe wszystkim miastom podług prawa sejmu teraźniejszego \textit{O miastach wolnych królewskich} zabezpieczamy. 
\item 3-tio. Sądy referendarskie \footnote{Sąd referendarski utworzony w 1507 r., najwyższy sąd dominialny króla.  Sąd jednostkowy. Rozpatrywał sprawy wniesione przez chłopów z królewszczyzn, ekonomii, dóbr klasztornych, należących do miast królewskich.} dla każdej prowincji osobne mieć chcemy w sprawach włościan wolnych, dawnymi prawami sądowi temu poddanych.  
\item 4-to. Sądy zadworne\footnote{Sądy zadworne - sądy dworu królewskiego ostatniej instancji, odbywające posiedzenia w miejscu pobytu króla.}, asesorskie\footnote{Sądy asesorskie - sądy królewskie apelacyjne, rozpatrujące odwołania od postanowień sądów miejskich.}, relacyjne\footnote{Sądy relacyjne - sądy asesorskie, w których zasiadał król z senatorami i ministrami. Rozpatrywanie spraw apelacyjnych z ziem lennych: Kurlandii i Prus Książęcych.} i kurlandzkie zachowujemy. 
\item 5-to. Komisje wykonawcze będą miały sądy w sprawach, do swej administracji należących.  
\item 6-to. Oprócz sądów w sprawach cywilnych i kryminalnych dla wszystkich stanów, będzie sąd najwyższy, sejmowy zwany, do którego przy otwarciu każdego sejmu obrane będą osoby. 
\end{itemize}
Do tego sądu należeć będą występki przeciwko narodowi i królowi, czyli crimina status. Nowy kodeks praw cywilnych i kryminalnych przez wyznaczone przez sejm osoby spisać rozkazujemy.  

 
\section*{IX. Regencja}

 Straż będzie oraz regencją, mając na czele królowę, albo w jej nieprzytomności prymasa.  W tych trzech tylko przypadkach miejsce mieć może regencja:  
\begin{itemize}
\item 1-mo w czasie małoletności króla; 
\item 2-do w czasie niemocy trwałe pomieszanie zmysłów sprawującej;  
\item 3-tio w przypadku gdyby król był wzięty na wojnie.  Małoletność trwać tylko będzie do lat 18 spełnionych;
\end{itemize}
a niemoc względem trwałego pomieszania zmysłów deklarowaną być nie może, tylko przez sejm gotowy większością wotów trzech części przeciwko czwartej izb złączonych.  W tych przeto trzech przypadkach prymas Korony Polskiej sejm natychmiast zwołać powinien, a gdyby prymas tę powinność zwłóczył, marszałek sejmowy listy okólne  do posłów i senatorów wyda.  Sejm gotowy urządzi kolej zasiadania ministrów w regencji i królową do zastąpienia króla w obowiązkach jego umocuje.  A gdy król w pierwszym przypadku z małoletności wyjdzie, w drugim, do zupełnego przyjdzie zdrowia, w trzecim z niewoli powróci, regencja rachunek z czynności  swoich oddać mu powinna i odpowiadać narodowi za czas swego urzędowania tak, jak jest przepisano o straży, na każdym ordynaryjnym sejmie,  z osób i majątków swoich.  

 
\section*{X. Edukacja dzieci królewskich}

Synowie królewscy, których do następstwa tronu konstytucja przeznacza, są pierwszemi dziećmi ojczyzny, przeto baczność o dobre ich wychowanie dla narodu należy, bez uwłoczenia jednak prawom rodzicielskim. Za rządu królewskiego sam król strażą i wyznaczonym od stanów dozorcą edukacji królewiczów wychowaniem ich zatrudniać się będzie. Za rządu regencji taż z wspomnianym dozorcą edukacji ich powierzoną mieć sobie będzie. W obydwu przypadkach dozorca, od stanów wyznaczony, donosić winien na każdym ordynaryjnym sejmie o edukacji i postępku królewiczów. Komisji zaś edukacyjnej powinnością będzie podać układ instrukcji i edukacji synów królewskich do potwierdzenia sejmowi, a to, jednostajne w wychowaniu ich prawidła wpajały ciągle i wcześnie w umysły przyszłych następców tronu religję, miłość cnoty, ojczyzny, wolności i konstytucji krajowej.
 
\section*{XI. Siła zbrojna narodowa}

 Naród winien jest sobie samemu obronę od napaści i dla przestrzegania całości swojej.  Wszyscy przeto obywatele są  obrońcami całości i swobód narodowych.  Wojsko nic innego nie jest, tylko wyciągnięta siła obronna i porządna z ogólnej siły narodu.  Naród winien wojsku swemu nadgrodę i poważanie za to, iż się poświęca jedynie dla jego obrony.  Wojsko  winno narodowi strzeżenia granic i spokojności powszechnej, słowem winno być jego najsilniejszą tarczą.  Aby przeznaczenia tego dopełniło niemylnie, powinno zostawać ciągle pod posłuszeństwem władzy wykonawczej, stosownie do opisów prawa, powinno wykonać przysięgę na wierność narodowi i królowi i na obronę konstytucji narodowej.  Użytem być więc wojsko narodowe może na ogólną kraju obronę, na strzeżenie fortec i granic, lub na pomoc prawu, gdyby kto egzekucji jego nie był posłusznym. 

\newpage
\pagenumbering{gobble}
\begin{itshape}
\noindent Kazimierz ksiażę Sapieha, generał artylerii lit., marszałek konfederacji W.Ks.Lit. \\
Stanisław Nałęcz Małachowski, referendarz w. kor., sejmowy i konfederacji prowincji koronnych marszałek. \\
Józef Korwin Kossakowski, biskup inflancki i kurlandzki, n. koadiutor biskupstwa wileńskiego jako deputowany. \\
Antoni książę Jabłonowski, kasztelan krakowski, deputat z senatu z Małej Polski. \\
Symeon Kazimierz Szydłowski, kasztelan żarnowski, deputowany z senatu z prowincji małopolskiej mp. \\
Franciszek Antoni na Kwilczu   Kwilecki, kasztelan kaliski, deputowany do konstytucji z senatu z prowincji   wielkopolskiej. \\
Kazimierz Konstanty Plater, kasztelan generału trockiego,   deputowany do konstytucji z senatu W. Ks. Lit. \\
mp. Walerian Stroynowski, podkomorzy buski, poseł wołyński, z Małopolski deputat do konstytucji. \\
Stanisław   Kostka Potocki, poseł lubelski, deputowany do konstytucji z prowincji małopolskiej \\
mp.   Jan Niepomucen Zboiński, poseł ziemi dobrzyńskiej, deputowany do konstytucji    z prowincji wielkopolskiej \\
mp. Tomasz Nowowieski, łowczy i poseł ziemi    wyszogrodzkiej, deputowany do konstytucji z prowincji wielkopolskiej.    \\
Józef Radzicki, podkomorzy i poseł ziemi zakroczymskiej, deputowany do    konstytucji z prowincji wielkopolskiej \\
mp. Józef Zabiełło, poseł z Księstwa    Żmudzkiego, deputowany do konstytucji. \\
Jacek Puttkamer, poseł województwa    mińskiego, deputowany do konstytucji z prowincji W. Ks. Lit.     \\
\end{itshape}

 
\end{document}
